\subsection{Controller Design}
In this subsection we discuss the method for designing the controller design parameters FeedbackGain-$K$ and FeedForwardGain-$F$. In this problem we have the scenario where $D_c<h$. Since our controller operates in discrete sample time, we start with converting our continuous system as described in Equation  \ref{eq:code1} into discrete domain. Therefore on applying ZOH sampling with period $h_c$ and constant sensor-actuator Delay $D_c$ we achieve the Equation  \ref{eq:code2} for discrete sample time system. From \ref{eq:code2} we can notice that the next output not only depends on current input but also on previous input. Hence, we simplify the system and get it into standard form representing Equation  \ref{eq:code3} by invoking Equation \ref{eq:code6}. After applying above simplification input $u[k]$ can be represented in terms of controller gains using Equation \ref{eq:code7}. 
and matrices $\Phi$ ,$C$ are converted to corresponding augmented matrices $\Phi_{aug}$,$C_{aug}$, whereas $\Gamma_1$,$\Gamma_2$ are converted to single augmented matrix $\Gamma_{aug}$.
\begin{equation}
\dot{x}  =  A x  +   B u  \quad{\text{   and   } }\quad  y  =  C x  
\label{eq:code1}
\end{equation}


\begin{equation}
x[k+1]= \phi x [k]+ \Gamma_1(D_c)u[k-1]+\Gamma_0(D_c)u[k] \quad \text{ and } \quad y[k] = C x[k]
\label{eq:code2}
\end{equation}


\begin{equation}
x[k+1]= \phi z [k]+ \Gamma_{aug}u[k] \quad \text{ and } \quad y[k] + C_{aug} z[k]
\label{eq:code3}
\end{equation}

\begin{equation}
z[k]=\begin{bmatrix}
x[k]\\
u[k-1]
\end{bmatrix}
\label{eq:code6}
\end{equation}

\begin{equation}
u[k] = Kz[k] + Fr 
\label{eq:code7} 
\end{equation}


\begin{equation}
 K = -[ \ 0 \ 0 \ \cdots \ 1 \ ] \gamma_{aug}^{-1} H(\phi_{aug})
\label{eq:code4} 
\end{equation}

\begin{equation}
 F = \dfrac{1}{C_{aug} (I-\phi_{aug}-\Gamma_{aug}K)^{-1}\Gamma_{aug}}
\label{eq:code5}
\end{equation}



  Before deriving the controller gains it is important to verify if the system is controllable under given configuration. This can be verified by calculating $det(\gamma_{aug})$. If the determinant is not equal to 0, then system is controllable. After verifying the controllability of the system $K$ can be derived by applying Ackermanns formula described in Equation \ref{eq:code4} and $F$ can be derived from Equation  \ref{eq:code5}. However, one can notice the matrix $H(\Phi_{aug})$ depends on the desired poles which in turn depends on the design requirements. The desired poles alter the frequency spectrum of the transfer function of the system and play a significant role in controlling the behavior of output parameters of the system. Therefore various design requirements such as Overshoot, settling time, boundary ranges of the parameters in the system depends on the desired poles and thus in turn influence controller gains $K$ and $F$. Following design work flow has been developed in MATLAB to derive an optimal configuration satisfying design constraints and maximizing cost function $\frac{QOC}{R}$.