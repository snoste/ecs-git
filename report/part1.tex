\chapter{Part 1}
%• REPORT – Part 1

\section{Introduction}
%• Very brief introduction of the overall problem setting (< page
%max)

\section{Response Time analysis}



\subsection{Response time analysis per processing unit}

% Table generated by Excel2LaTeX from sheet 'Sheet1'
\begin{table}[htbp]
	\centering
	\caption{By running the Matlab script ResponsetimeAnylsis\_FPP.m with the different parameters given for PU1 and PU2 these response times are obtained. These files are then delivered as PU1.m PU2.m}
	\begin{tabular}{rrrrr}
		\toprule
		PU1     & $T_1$    & $T_2$    & $T_3$    & $T_4$  $(T_s)$ \\
		\midrule
		ms      & 0.1     & 2.1     & 4.1     & 7.2 \\
		&         &         &         &  \\
		\toprule
		PU2     & $T_5$    & $T_6$    & $T_7$    & $T_8$ \\
		\midrule
		ms      & 6       & 3       & 9       & 5 \\
		
	\end{tabular}%
	\label{tab:addlabel}%
\end{table}%


\subsection{ Response time analysis for the CAN bus messages}

\section{System model}
%• System model derivation, design space exploration and controller
%parameter design

\section{Design decision}
%• Your design decision and justification.

\section{Results}
%• Results
Firstly: Response time analysis\\
Secondly: Control system input and output









%• 3 processors:
%• PU1: Fixed Priority Preemptive (FPP)
%• PU2: Fixed Priority Preemptive (FPP)
%• PU3: CompSOC platform using Time Division Multiplexing (TDM) with a clock
%frequency running @ 100 MHz
%− Partition slots: 245904 clock cycles: 2.45904 ms
%− CoMik slots: 4096 clock cycles: 40.96 us
%− TDM period: 1000000 clock cycles: 10 ms
%− 4 Partition slots, 4 CoMik slots
%• Multiple tasks running on each processor
%• PU1: T1, T2, T3, Ts
%• PU2: T5, T6, T7, T8
%• PU3: S1 (unassigned), S2 (T4), S3 (Tca), S4 (unassigned)
%• Communication task in all processors has a WCETcom = 0.05 ms
%• Communication bus : CAN
%• Transmission of each message over the CAN bus takes 1 ms

%
%• Design the controller and its implementation such that
%1. The real-time tasks meet their deadlines and end-to-end delay
%requirements. For this assign the priorities in each processor
%and the CAN bus to meet the requirements. You may use the
%response time analysis MATLAB scripts provided with the
%assignment.
%2. Input signal 𝐕𝐕 should be designed such that the motor position
%(shatft angle) is 𝜽𝜽 = 𝟑𝟑. 𝟏𝟏𝟏𝟏𝟏𝟏𝟏𝟏 rad for a maximum available input
%signal is 1 V, and maximum settling time of 400 ms
%3. A design space exploration of system poles which satisfy the
%requirements (input signal, settling time) should be done.
%4. Hint: use the sampled-data model used in the lectures (derive
%the augmented system). Be careful during discretization.

%• Response time analysis processor PU1 (PU1.m)
%• Response time analysis processor PU2 (PU2.m)
%• Response time analysis for CAN messages (CAN.m)
%• Control design and simulation (Control_p1.m)
%• A PDF report (a2.pdf)
%• See details of the report later